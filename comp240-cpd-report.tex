% Please do not change the document class
\documentclass{scrartcl}

% Please do not change these packages
\usepackage[hidelinks]{hyperref}
\usepackage[none]{hyphenat}
\usepackage{setspace}
\usepackage{url}
\doublespace

\title{COMP230 - CPD Report}
\author{1706165}

\begin{document}

\maketitle

\section{Introduction}
In this update of my on-going Continual Personal Development, I will try to raise a few things that I need to work on as an individual to bolster my chances of becoming a well-rounded individual and hopefully securing a financially safe future. To tackle these issues I will be using S.M.A.R.T targets to help solidify what I need to do to assure these issues don't happen again. In this report, I will be covering Time-Management, Approachability, Communication, Attendance and Contribution.


\section{Time-Management and Attendance}
During this study-block, I found myself struggling to actually attend the sessions on my timetable due to my own neglect and lack of desire for University as a whole. Currently, I live about 30-minutes away from the campus which means that to even attend a session, I'd have to plan an hour before to even attempt to catch a bus. This was a problem for me because it hindered my desire to learn and complete coursework because I couldn't manage my time to wake up and go and catch the bus every morning. Accumulating missing days means that more days are missed as a result and overall, this is just a bad habit to have as it affects coursework in the long-run.

Action: Make sure to set an early alarm every morning at 7:00 or 8:00 AM to make sure that I can get into University on time to attend my sessions. Start an organizer that has the lectures for the day and tick off each day as it comes. 


\section{Approachability/Communication}
I didn't think this was an issue to start with but I have started to notice that people tend to avoid others that may look slightly intimidating or different. This affects the team dynamics as some team-members may avoid talking to you but will instead talk about you and within a professional setting not recognizing issues as issues may cause conflict in the long run. As such, body language and general politeness go a long way as I have come to learn this year. However, I have also learned that you cannot change people and the way they think of you unless both parties are willing to do so. For example; this year, a fellow programmer has stripped a task away from me without my consent and I found that to be extremely disrespectful in the manner it was performed. I have tried to make sure this isn't an issue but the other person just has set opinions of me that I just don't care to change. 

Action: Make an effort towards changing my body language to be less intimidating and be open towards talking to people that I wouldn't usually talk to. 


\section{Contribution}
This study block, I presented ideas that I think would suit the game very well and these ideas were the Spinning blade and the Swinging blade. It was refreshing to work on your own contributions but after these tasks were done, I found myself having nothing to actually work on or contribute to the game. This is an issue because if I could have communicated more ideas after these two were done, the game may have turned out very differently. I took the easy way out by not doing anything when in reality, I could have spent that time working on an interesting new mechanic that would change how the game is played entirely. Instead I spent that time being upset at the fact that my own task was took away from me because I didn't finish it in time. After that point, my motivation for the project dropped marginally as I just didn't think of the game as my own any more and as such, I couldn't think of anything else to contribute to the project. Next time, I will make sure that my tasks are always on-going and never stalled.

Action: Make sure to update your team on the tasks that are on-going. In addition, scope the features that you can put in earlier in the development cycle so then the team will figure out
what is within your capabilities and what isn't.

\section{Public Speaking}
Presentations themselves have never been my strong suit. Anxiousness before and nervousness during a presentation is something that I have to deal with each time. Although may be not noticeable but the pressure is sometimes daunting on me and I tend to try and get through everything as fast as possible, this is reflected in my speech as my main goal is to be done with it as soon as possible. I have done multiple presentations up to this point and have improved in terms of being able
to get it done whereas before I would just crumble. Being comfortable with speaking in
front of people will greatly improve my pitches of games and feeling relaxed when doing
a presentation will assure that I am still getting through the content but doing it at my
own pace without fast forwarding. Leading a group of 10+ people, if I am not confident
in public speaking and unclear with my points, it could scrutinize the whole project as
everyone would follow their own perspective if my ideas are misunderstood.I need to prepare for a presentation properly via written cards and make sure to prepare content to then present as well as constant practice in front of other people to make it seem less intimidating.

Action: Get more comfortable talking and presenting ideas to people i.e. attending a group every week for an hour. 


\section{Conclusion}
In conclusion, this study-block was a time of many blunders but compared to the first study-block, this was nothing. In contrast, my attendance compared to the first study-block has fallen significantly. This was largely due to just burning out completely and having issues at home to deal with at the same time. It can get hectic to keep up with especially with deadlines being dropped like live ammunition shells. Overall, as I am progressing as an individual, it has become easier to work in a team as I can compensate for my own flaws by recognizing them as such and making sure to avoid making the same mistakes over and over again.


\end{document}
